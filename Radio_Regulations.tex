\documentclass[12pt,titlepage,openbib]{beamer}

\author{Pritvi Jheengut @zcoldplayer}
\title{Radio Regulations for IOT in Mauritius}
%\subtitle{}

% \usepackage{ucs}
\usepackage[T1]{fontenc}
\usepackage[utf8]{inputenc}
\usepackage[size=a4, orientation=landscape, scale=4]{beamerposter}

\usepackage{lmodern}
\usepackage{fontenc}[T1]
\usepackage{inputenc}[utf8]
\usepackage{hyperref}[pdfpagelabels=true,bookmarks=true,388unicode=true]

% \usepackage[pdftex]{hyperref}
% \hypersetup{%
%   bookmarks=,%
%   pdfcreator=,%
%   pdffitwindow=%
% }

\usepackage{pgf}
\usepackage{verbatim}

\usepackage{listings}
\usepackage{xy}[all]
% \usepackage[english,francais,UKenglish]{babel}

% Eliminate errors such as
% LaTeX Font Warning: Font shape `T1/cmss/m/n' in size <4> not available
% LaTeX Font Warning: Size substitutions with differences up to 1.0pt 

% \usepackage[pdftex]{graphicx}
\usepackage{graphicx}
\usepackage{booktabs}

\mode<presentation>{

  \usetheme{Frankfurt}
  \useoutertheme{infolines}
  \usecolortheme{cormorant}
  
  \definecolor{darkblue}{HTML}{100080}
  
  \setbeamercolor{Title bar}{parent=black,bg=darkblue,fg=white}
  \setbeamercolor{title}{parent=black,bg=darkblue,fg=white}
  \setbeamercolor{frametitle}{parent=white,bg=darkblue,fg=gray}
  \setbeamercolor{titlelike}{parent=gray,bg=darkblue,fg=black}
  \setbeamercolor{block title}{parent=black,bg=darkblue,fg=gray}
  
  \setbeamercovered{transparent,dynamic}
  
  \setbeamertemplate{blocks}[rounded][shadow=true]
  \setbeamertemplate{background canvas}[vertical shading]
                    [top=darkblue!80,middle=darkblue!50,bottom=darkblue!80]
                    
% \mode<handout>{\beamertemplatesolidbackgroundcolor{black!50}}
                    
}

\makeindex

\begin{document}

\date[]{Phase II \\
  University of Mauritius \\
  Saturday 15$^{th}$ February 2020}

\maketitle

\tableofcontents

\section{CopyLeft License}

\frame{
  \frametitle{Copyleft License Attribution}
  
  Made with love using beamer, LaTeX and git.\\
  You can view at \href{https://gitlab.com/Presentation-PritviJheengut/Radio_Regulations_for_IOT_in_Mauritius
  }{Radio Regulations for IOT in Mauritius}
  
  \begin{alertblock}{This work is licensed under the LaTeX Project
      Public License.}

    To view a copy of this license, visit \\
    \url{https://www.latex-project.org/lppl.txt} 
    
  \end{alertblock}
  
  \begin{alertblock}{This work is licensed under the Creative
      Commons Attribution 4.0 International License.}

    To view a copy of this license, visit \\
    \url{http://creativecommons.org/licenses/by/4.0/} or \\
    send a letter to \\
    Creative Commons, \\
    PO Box 1866, \\
    Mountain View, \\
    CA 94042, \\
    USA. \\

  \end{alertblock}
}

\section{Greetings}

\frame{
  \frametitle{Who Am I}
  
  \begin{block}{Who Am I}
    
    \href{http://slackware.com/}{Geek@Slackware}
    
    \href{https://twitter.com/zcoldplayer}{twitter @zcoldplayer}
    
    \href{https://xmail.net/z.coldplayer}{zcoldplayer xmail Website}
    
    \href{http://metservice.intnet.mu/}{Work
      : SMTT@Meteorological.Services.mu}
    
    Active in many User group LUGM, MMC, MSCC, FECM, GDG\_MU \\
    and several other hackathons
    
    Passionate about how and why things work.
    
    Fervour Advocate of Free Libre and Open Source Software.
    
  \end{block}

}

\frame{
  \frametitle{Who are you}
  
  \begin{block}{Would you mind tell me who you are?}

    \begin{enumerate}
      
    \item @twitter\_handle
    \item where you work
    \item email you want to share
    \item Hobbies
    \item purpose and expectations of this session

    \end{enumerate}      

  \end{block}
  
}

\section{Community Groups in Mauritius}

\frame{
  \frametitle{Community Groups in Mauritius}
  
  \begin{block}{Healthy growth of Community Groups in Mauritius}
    
    This turn of the century has seen an uprising of Community
    groups in Mauritius in the field of the Digital World. The
    diversity has helped the exchange and sharing of innovative
    ideas, experience, bleeding edge technology, upcoming events,
    conferences in the Digital Island of the Republic of
    Mauritius.

  \end{block}
  
  This is a list of some of the active communities in Digital
  Mauritius.

  \begin{itemize}
    
  \item Linux User Group Meta, LUGM
  \item Mauritius Software Craftmanship Community, MSCC
  \item Mauritius Makers Community, MMC
  \item Front-End Coders Mauritius, FECM
  \item PHP User Group of Mauritius, phpMauritiusUG
  \item Symfonymu
  \item Google Developers Group Mauritius, GDG\_M
  \item Digital Marketing Mauritius
  \item Python User Group, PyMUG
    
  \end{itemize}
  
}

\section{How I relate to this subject}

\frame{
  \frametitle{My experience with Radio Regulations}

  \begin{block}{My experience with Radio Regulations}

    \large
    
    \begin{itemize}
    \item My project was Software Correlation for Radio Astronomy at
      The Mauritius Radio Telescope, MRT.
    \item I work at the Meteorological Services and I work with both VHF
      and HF Radio systems as well as the Doppler Weather Radar.
    \item \href{https://en.wikipedia.org/wiki/DVB-T}{I use a DVB-T
      reciever unit.}
    \item Co-founder of the Mauritius Maker Community
    \end{itemize}

  \end{block}
  
}

\section{IOT and Radio Communication}

\frame{
  \frametitle{What is IoT?}

  \begin{block}{What is IoT}

    \begin{center}

      Internet of Things
      
      Always internet connected computing device
      
      \pause
      
      eg :: mobile phone

      I'll continue to use this as a reference for IoT
      and radio frequencies.
      
    \end{center}

  \end{block}

  \pause

  \begin{block}{Communication for IoT}

    Communication for IoT can either be wired or wireless or
    both at the same time.

    In this session, it will be assuming that wireless communication
    has been deployed for IoT.    

  \end{block}

  \begin{block}{Mobile Phones}

    As an IoT, mobile phones can always connected to the internet
    using several distinct wireless technologies.
    
  \end{block}

}

\frame{
  \frametitle{Radio Waves}

  \begin{block}{Radio frequency waves}

    has been the most commonly used way of carrying data wirelessly
    since more than a century after Hertz proved that electromagnetic
    waves can be used to carry a signal.

  \end{block}

  \pause
  
  \begin{block}{Since then a lot has changed and much of the Radio Spectrum}

    is utilised for several \href{https://www.youtube.com/watch?v=u_aLESDql1U
    }{billions and billions} of frequencies.

  \end{block}
  
  \begin{alertblock}{Radio Spectrum}
    
    \href{https://upload.wikimedia.org/wikipedia/commons/d/df/United_States_Frequency_Allocations_Chart_2011_-_The_Radio_Spectrum.pdf}{US Radio Spectrum PDF}

  \end{alertblock}
  
}

\frame{
  \frametitle{Radio Spectrum}

  \begin{example}

    Mobile Digital Phones are always connected to

      \begin{itemize}

      \item \href{https://en.wikipedia.org/wiki/GSM}{GSM}
        - Global System for Mobile Communications
      \item \href{https://en.wikipedia.org/wiki/General_Packet_Radio_Service}{
        GPRS} - General Packet Radio Service
      \item \href{https://en.wikipedia.org/wiki/3G}{3G} 
      \item \href{https://en.wikipedia.org/wiki/LTE_Advanced}{4G/LTE}
      \item \href{https://en.wikipedia.org/wiki/5G}{5G}
        
      \end{itemize}
      
  \end{example}
  
  \pause
  
  \begin{example}
    
    \href{https://en.wikipedia.org/wiki/DVB-T}{DVB-T has replaced analog TV}
    
  \end{example}
  
  \pause
  
  \begin{example}
    
    \href{https://en.wikipedia.org/wiki/Radio_spectrum
    }{Electromagnetic Radio Spectrum}
    
  \end{example}
  
}

\frame{
  \frametitle{Radio Frequency Issues}
  
  \begin{block}{Radio Frequency Electromagnetic waves}
    
    \begin{center}
      
      entails many issues such as ::
      
      \begin{itemize}
        
      \item Interference
      \item Acceptable safe Radiation levels  
      \item Electromagnetic compatibility 
      \item Security
        
      \end{itemize}
      
    \end{center}
    
  \end{block}
  
  \pause
  
  \begin{block}{Solution}
    
    \begin{center}
      
      \begin{itemize}
        
      \item Compliance with Standards
      \item Certification 
      \item Spectrum Management
      \item Safety Normalisations
        
      \end{itemize}  
      
    \end{center}
    
  \end{block}
  
  \begin{alertblock}{Radio Spectrum}
    
    \begin{center}
      
      \href{https://upload.wikimedia.org/wikipedia/commons/d/df/United_States_Frequency_Allocations_Chart_2011_-_The_Radio_Spectrum.pdf}{US Radio Spectrum PDF}
      
    \end{center}
    
  \end{alertblock}
    
}

\section{Regulations in Mauritius}

\frame{
  \frametitle{ICTA}
  
  \begin{block}{\href{https://www.icta.mu/about.html}{ICTA.mu}}
    
    is the regulatory body in charge of

    \begin{itemize}
      
    \item Allocation
    \item Management
    \item Review
    \item Organisation
    \item Regulation
      
    \end{itemize}

    of the frequency spectrum in Mauritius by setting up a radio
    frequency management unit and be the Controller of
    Certification Authorities.
    
  \end{block}
  
  Nonetheless, the ICTA implements Government policy and provide
  technical monitoring of the ICT in accordance with recognised
  international standard practices, protocols.
    
}

\frame{
  \frametitle{ICTA Regulations}

  \large
  
  \begin{block}{\href{https://www.icta.mu/docs/laws/Clearance_Regulations2019.pdf
      }{ICTA Clearance Regulations}}
    
    Information on regulations and clearance to Import ICT Equipment
    such as CE marking means the marking affixed on the packaging.
    
  \end{block}

  \begin{definition}
    
    ICT equipment means an equipment intended for telecommunication
    or radio communication.
    
  \end{definition}

}

\frame{
  \frametitle{Some examples of Radio Licenses}

  \begin{block}{Types of Licenses}

    you can apply for in Mauritius
      
      \begin{itemize}
        
      \item Radio Apparatus Licence
      \item Radio Amateur
      \item INMARSAT Mobile Earth Station
      \item Network Spectrum
      \item Temporary Test Licence for Frequency Usage
      \item Fixed Radio Spectrum
        
      \end{itemize}
  
  \end{block}
  
  \begin{block}{Activities of ICTA includes}
      
      \begin{itemize}
        
      \item Spectrum assignment and licensing
      \item Station and Apparatus licensing
      \item International Spectrum coordination
        
      \end{itemize}
      
  \end{block}
  
}

\section{International Bodies}

\frame{
  \frametitle{ICTA and International Bodies}
  
  \begin{block}{The ICTA ensures operation within the limits of radio frequency}

    radiation as recommended by the \href{https://www.icnirp.org}{International
      Commission on Non-ionizing Radiation Protection, ICNIRP}.
    
  \end{block}
  
  \begin{block}{Other International Bodies include}
      
      \begin{itemize}
        
      \item relevant ITU-R Study Groups
      \item ITU Radio Assemblies 
      \item Regional Radiocommunication Conferences
      \item World Radio Conferences 
        
      \end{itemize}

  \end{block}

}

\section{The End and Questions}

\frame{
  \frametitle{Questions}
  
  \huge
  
  \begin{block}
    
    QUESTIONS!
    
  \end{block}
  
}

\frame{
  \frametitle{Questions}

  \huge

  \begin{block}

    Thanks for your Time

    Twitter \href{https://twitter.com/zcoldplayer}{@zcoldplayer}

    or

    Email \href{https://xmail.net/z.coldplayer}{z dot coldplayer at
      xmail dot net}

  \end{block}

}

\end{document}
